\section{Introduction}

Nested data structures usually impose a performance and memory management challenge.
A straightforward implementation allocates memory for each level of such data structures, leading to situations where the components of a data structure are spread out over a wide region of memory.
This introduces several challenges in terms of runtime efficiency.
Many small allocations tax memory management and reference counting, and bulk operations on such data suffer from poor memory locality.

Functional programming languages such as Haskell, Clean, and ML all use algebraic data types as one of their main ways of constructing data.
To keep the memory management overhead at bay, these languages typically use garbage collection schemes like mark-and-sweep~\cite{mark-and-sweep} or more sophisticated versions thereof such as generational garbage collection~\cite{generational-gc}.
More recently, work in the context of Koka~\cite{koka,kokamem} tackles the memory management overhead by statically identifying reuse opportunities.
While these techniques help deal with the memory management overhead, they usually do not improve memory locality.

Dealing with locality, in particular when aiming at bulk-synchronous parallel operations, is most effective for data that is stored in a flat way in memory.
To achieve this, arrays are better suited: they lead to larger grain allocations and they allow for better data locality through the use of a flattened notation instead of a nested one.
To further improve these aspects when creating nested arrays, performance oriented languages typically apply some form of flattening.
Languages such as Accelerate~\cite{accelerate}, Futhark~\cite{futhark} and SaC~\cite{sac,sac2} aim at purely flattened memory representations for nested arrays at runtime.

Once we add records to these languages, we run the risk of losing the advantages of a flattened representation, in particular when creating several layers of nested structures such as arrays of records that contain further
arrays within their fields.
A naive implementation would result in a large amount of small allocations, not to mention the loss of locality when accessing elements of fields in neighbouring records.
Instead, we would like to reduce the number of allocations to only a single allocation for each field.
The key idea is to enable flattening ``through'' records.
To achieve this, we rewrite records into separate arrays; a vector of records that contain two vectors each thus turns into two two-dimensional arrays.
Not only does this resolve the overhead due to allocations and the consequent loss of locality, it also implies that we no longer need to support records at runtime at all.
The record extension that is described in this paper works for all uses of non-recursive records and in which all fields have known size.
The extension has been implemented in SaC as of version 1.3.3-1175-1 and can be downloaded from \url{https://sac-home.org/download:main}.

The contributions of this paper are:
\begin{itemize}
    \item A syntax for records, constructors and accessors that introduces immutable records while preserving compliance with an imperative specificational style.
    \item A transformation scheme that translates programs that operate on potentially nested record structures into programs that do without.
    \item Optimisations for eliding unused arguments and unmodified return values from function applications, respectively, and
    \item A real-world example showing how the presented techniques cooperatively transform nested arrays of records into record-free multi-dimen\-sional arrays.
\end{itemize}
