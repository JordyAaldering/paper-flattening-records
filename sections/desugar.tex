\subsection{Constructors, Accessors, and Mutators}

The first step in transforming programs with records into programs without records is to replace record constructors, accessors, and mutators by generated function definitions.
For each record type in a program we generate a full constructor and a default constructor function, as well as accessor and mutator functions for every field of that record.
Any record constructors, accessors, and mutators in the program are then replaced by applications of the corresponding generated functions.

\subsubsection{Accessors and Mutators}

Normally an infix dot symbol is used to access or mutate the field of a record type, e.g. \texttt{bodies.pos}.
In order to access or mutate a nested field, multiple accessors or mutators may be chained.
However after programs have been transformed there will no longer be any records, and such a selection is no longer applicable.
Instead we generate accessor and mutator functions for every field occurring in a record type.
To access and mutate the position field of the body record, for example, these would be \verb|body_set_pos| and \verb|body_get_pos| respectively.
For a record type with $n$ fields, we generate an accessor and a mutator function for every field $i\in[1,n]$:
%
\begin{lstlisting}[escapechar=$]
type$\subb{1}$[shp$\subb{1}$], ..., type$\subb{n}$[shp$\subb{n}$]
rt_get_id$\sub{i}$(type$\subb{1}$[shp$\subb{1}$] id$\sub{1}$, ..., typen[shp$\subb{n}$] id$\sub{n}$)
{
    return id$\sub{i}$;
}

type$\subb{1}$[shp$\subb{1}$], ..., type$\subb{n}$[shp$\subb{n}$]
rt_set_id$\sub{i}$(type$\subb{i}$[shp$\subb{i}$] value, type$\subb{1}$[shp$\subb{1}$] id$\sub{1}$, ..., type$\subb{n}$[shp$\subb{n}$] id$\sub{n}$)
{
    return (id$\sub{1}$, ..., id$\sub{i-1}$, value, id$\sub{i+1}$, ..., id$\sub{n}$);
}
\end{lstlisting}
%
Note that the argument and return types in these generated functions can still be records at this point.
We expand these records at a later step, along with the expansion of records in user-defined functions.

Accessors and mutators through field selection can now be replaced by applications of these generated functions.
For a field selection \verb|id = x.y| on the right-hand-side of a let-expression we apply the accessor function \verb|id = rt_get_y(x)|, whereas if on the left-hand-side there occurs a field selection \verb|x.y = expr|, the mutator function is applied \verb|rt_set_y(expr, x)|.

\subsubsection{Constructors}

Syntactically there are three kinds of record constructors:
a full constructor that expects a value for each field in order,
a default constructor that takes no arguments and assigns a default (zero) value to each field,
and an explicit constructor with only the values for some fields explicitly defined.
For every record type \texttt{rt} in the program we generate a new function definition for both the full and the default constructor:
%
\begin{lstlisting}[escapechar=$]
type$\subb{1}$[shp$\subb{1}$], ..., type$\subb{n}$[shp$\subb{n}$]
new_rt(type$\subb{1}$[shp$\subb{1}$] id$\sub{1}$, ..., type$\subb{n}$[shp$\subb{n}$] id$\sub{n}$)
{
    return (id$\sub{1}$, ..., id$\sub{n}$);
}

type$\subb{1}$[shp$\subb{1}$], ..., type$\subb{n}$[shp$\subb{n}$]
zero(type$\subb{1}$[*] id$\sub{1}$, ..., type$\subb{n}$[*] id$\sub{n}$)
{
    return new_rt(genarray([shp$\sub{1}$], zero([:type$\subb{1}$])
                  ...
                  genarray([shp$\sub{n}$], zero([:type$\subb{n}$]));
}
\end{lstlisting}
%
Where \texttt{[:type]} is an empty array of the given type.
This ensures that the correct overload of the \texttt{zero} function is applied, returning the default value for that type.

Now whenever we encounter a full constructor of the form \texttt{rt\{expr\sub{1}, ..., expr\sub{n}\}} we replace it by \texttt{new\_rt(expr\sub{1}, ..., expr\sub{n})}, and whenever we encounter a default constructor of the form \texttt{rt\{\}} we replace it by \texttt{zero([:rt])}.
Finally, if we encounter an explicit constructor of the form
%
\begin{lstlisting}[escapechar=$]
rt{.field$\sub{q}$ = value$\sub{q}$, ..., .field$\sub{r}$ = value$\sub{r}$}
\end{lstlisting}
%
we first apply the default constructor, followed by a chain of mutator functions for the explicitly given fields.
%
\begin{lstlisting}[escapechar=$]
rt_set_id$\sub{r}$(value$\sub{r}$,
    ...
        rt_set_id$\sub{q}$(value$\sub{q}$, zero([:rt]));
\end{lstlisting}