\section{Example}\label{sec:example}

As a running real-world example we consider the \textit{naive n-body problem}, which is concerned with predicting the individual motions of a group of celestial bodies.
We define a record \texttt{Body} to keep track of these celestial bodies, as well as a \texttt{Vector3} record for the position of this body.
Using these we define a function that computes the acceleration one body imposes on another, along with a function that computes the acceleration imposed on each body by all other bodies.
%
\begin{lstlisting}
struct Vector3 {
    double x;
    double y;
    double z;
};

struct Body {
    struct Vector3 pos;
    double mass;
};

struct Vector3
acc(struct Body b1, struct Body b2)
{
    dir = b2.pos - b1.pos;
    factor = l2norm(dir) == 0.0
        ? 0.0 : b2.mass / pow3(l2norm(dir));
    return factor * dir;
}

struct Body[n]
acc_v(struct Body[n] bodies)
{
    return { [i] -> rsum(1, { [j] -> acc(bodies[i], bodies[j])
                            | [j] < shape(bodies) })
           | [i] < shape(bodies) };
}
\end{lstlisting}
%
With these acceleration functions we can define a function that computes the updated positions of all bodies after some given delta time \texttt{dt}.
For this each body additionally requires a velocity field, which we can add to the record without requiring modifications to the acceleration function, highlighting one of the benefits of using records.
%
\begin{lstlisting}
struct Body {
    struct Vector3 pos;
    double[3] vel;
    double mass;
};

struct Body[n]
timestep(struct Body[n] bodies, double dt)
{
    acceleration = acc_v(bodies);
    bodies.pos += bodies.vel * dt;
    bodies.vel += acceleration * dt;
    return bodies;
}
\end{lstlisting}
